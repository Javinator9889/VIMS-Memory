\begin{acronym}
  \acro{ADAS}{\textit{Advanced Driver Assistance Systems}}
  \acro{ALDL}{\textit{Computerized Assembly Line Diagnostics Link}}
  \acro{API}{\textit{Application Programming Interface}}
  \acro{BLE}{\textit{Bluetooth Low Energy}}
  \acro{CAN}{\textit{Controller Area Network}}
  \acro{CARB}{\textit{California Air Resources Board}}
  \acro{DTC}{\textit{Diagnostic Trouble Codes}}
  \acro{ECU}{\textit{Engine Control Unit}}
  \acro{EOBD}{\textit{European On-Board Diagnostics}}
  \acro{GB}{\textit{gigabyte}}
  \acro{GPS}{\textit{Global Positioning System}}
  \acro{GUI}{\textit{Graphical User Interface}}
  \acro{IoT}{\textit{Internet of Things}}
  \acro{JOBD}{\textit{Japanese On-Board Diagnostics}}
  \acro{LTE}{\textit{Long--Term Evolution}}
  \acro{MIL}{\textit{Malfunction Indicator Lamp}}
  \acro{MSB}{\textit{Most Significant Bit}}
  \acro{OBD}{\textit{On-Board Diagnostics}}
  \acro{PAN}{\textit{Personal Area Network}}
  \acro{PFM}{Proyecto Fin de Máster}
  \acro{PID}{\textit{Parameter ID}}
  \acro{RPM}{Revoluciones Por Minuto}
  \acro{SAE}{\textit{Society of Automotive Engineers}}
  \acro{UART}{\textit{Universal Ascynchronous Receiver--Transmitter}}
  \acro{VIMS}{\textit{Vehicle IoT Metrics System}}
  \acro{VIN}{\textit{Vehicle Identification Number}}
\end{acronym}

\begin{itemize}
  \item \ac{ADAS} -- sistema de control de un vehículo que utiliza sensores del entorno
        (radar, láser, visión por computador) para mejorar la seguridad al volante y
        la seguridad del entorno ayudando a los conductores a reconocer y reaccionar
        ante eventos o situaciones de tráfico potencialmente peligrosas \cite{hermawanAcquisitionModelingEvaluating2020}.
  \item \ac{API} -- conjunto de definiciones, subrutinas y protocolos que ofrecen
        ciertos \textit{softwares} para ser usados por otras aplicaciones como
        capa de abstracción sobre el original \cite{InterfazProgramacionAplicaciones2021}.
  \item \ac{BLE} -- tecnología \ac{PAN} que permite la comunicación entre dispositivos
        con un rango similar a Bluetooth pero un menor consumo de energía.
  \item Bus \ac{CAN} -- protocolo de comunicaciones de tiempo real para el envío de
        mensajes en entornos distribuidos, permitiendo la comunicación
        entre múltiples CPUs.
        misma red o malla. Tiene un alcance muy limitado, de unos pocos metros por
        lo general.
  \item \ac{DTC} -- códigos de diagnóstico almacenados en un vehículo que identifican
        un error que tiene el vehículo o que ha tenido.
  \item \ac{ECU} -- centralita electrónica conectado a todos los sensores y sistemas
        del vehículo que recibe la información de los mismos y ejecuta comandos o acciones
        contra ellos.
  \item \ac{GB} -- unidad de almacenamiento de información equivalente a $10^9$ bytes.
  \item \ac{GPS} -- sistema que permite posicionar cualquier objeto con una 
        precisión de hasta centímetros usando cuatro o más satélites y 
        trilateración \cite{GPS2021}.
  \item \ac{GUI} -- conjunto de componentes visuales que permiten a los usuarios
        interactuar con una aplicación de manera visual e intuitiva.
  \item \ac{IoT} -- concepto que se refiere a la interconexión digital de objetos 
        cotidianos con Internet \cite{InternetCosas2021}.
  \item \textit{jitter} -- retardo relativo que se produce en las comunicaciones
        y que afecta directamente a la saturación de la red y a la capacidad de
        transmisión de la misma.
  \item \ac{LTE} -- también conocido como ``\texttt{4G}'', es la cuarta generación
        del estándar de conexión móvil a Internet que mejora a su predecesor, el \texttt{3G}.
  \item \ac{MSB} -- bit, de acuerdo con su posición, tiene mayor valor (extremo izquierdo).
  \item \ac{OBD} -- sistema de diagnóstico a bordo de vehículos que
  \item \ac{PAN} -- redes destinadas a la comunicación entre dispositivos en una
        cuenta con múltiples estándares según la región de uso. Estos
        sistemas ofrecen una monitorización activa y control completo
        sobre el motor y otros dispositivos del vehículo \cite{OBD2021}.
  \item \ac{PID} -- parámetro utilizado en la comunicación mediante \ac{OBD}--II que
        identifica el tipo de petición que el mecánico/usuario quiera hacer al sistema.
        Son valores enteros de 16 bits que funcionan en formato clave--valor. La información
        devuelta por el conector es también un conjunto de bytes de 32 bits en total que
        representan la información solicitada por el usuario. Por lo general, se deben
        realizar ciertas operaciones aritméticas con ese conjunto de bytes para obtener
        la representación decimal del valor \cite{OBDIIPIDs2021}.
  \item \textit{Smog} -- nube baja formada por dióxido de carbono, hollín, humo y polvo
        en suspensión que se forma sobre las grandes ciudades o núcleos industriales.
  \item trilateración -- método matemático que permite determinar las posiciones
        relativas de objetos usando la geometría de los triángulos \cite{Trilateracion2021}.
  \item \ac{VIN} -- código único, incluyendo el número de serie, que identifica individualmente
        los vehículos a motor, a remolque, motocicletas, \textit{scooters} y \textit{mopeds},
        tal como se define en las ISO 3779 (contenido y estructura) e ISO 4030 (ubicación
        y acoplamiento) \cite{VehicleIdentificationNumber2022}.
\end{itemize}
