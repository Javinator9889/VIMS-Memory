\begin{acronym}
  \acro{ADAS}{\textit{Advanced Driver Assistance Systems}}
  \acro{ADC}{\textit{Analog--Digital Converter}}
  \acro{ALDL}{\textit{Computerized Assembly Line Diagnostics Link}}
  \acro{API}{\textit{Application Programming Interface}}
  \acro{BLE}{\textit{Bluetooth Low Energy}}
  \acro{BR/EDR}{\textit{Basic Rate/Enhaced Data Rate}}
  \acro{CAN}{\textit{Controller Area Network}}
  \acro{CARB}{\textit{California Air Resources Board}}
  \acro{CRUD}{\textit{Create Read Update Delete}}
  \acro{DMS}{\textit{Deadline Monotolic Scheduling}}
  \acro{DTC}{\textit{Diagnostic Trouble Codes}}
  \acro{ECU}{\textit{Engine Control Unit}}
  \acro{EOBD}{\textit{European On-Board Diagnostics}}
  \acro{GB}{\textit{gigabyte}}
  \acro{GPIO}{\textit{General Purpose Input/Output}}
  \acro{GPS}{\textit{Global Positioning System}}
  \acro{GUI}{\textit{Graphical User Interface}}
  \acro{ICPP}{\textit{Immediate Ceiling Priority Protocol}}
  \acro{IoT}{\textit{Internet of Things}}
  \acro{JOBD}{\textit{Japanese On-Board Diagnostics}}
  \acro{LTE}{\textit{Long--Term Evolution}}
  \acro{MIL}{\textit{Malfunction Indicator Lamp}}
  \acro{MSB}{\textit{Most Significant Bit}}
  \acro{NTP}{\textit{Network Time Protocol}}
  \acro{OBD}{\textit{On-Board Diagnostics}}
  \acro{P2P}{\textit{Point-to-Point}}
  \acro{PAN}{\textit{Personal Area Network}}
  \acro{PFM}{Proyecto Fin de Máster}
  \acro{PID}{\textit{Parameter ID}}
  \acro{PWM}{\textit{Pulse--Width Modulation}}
  \acro{REST}{\textit{REpresentational State Transfer}}
  \acro{RPM}{Revoluciones Por Minuto}
  \acro{SAE}{\textit{Society of Automotive Engineers}}
  \acro{SoC}{\textit{System on Chip}}
  \acro{UART}{\textit{Universal Ascynchronous Receiver--Transmitter}}
  \acro{ULP}{\textit{Ultra Low Power}}
  \acro{UTC}{\textit{Coordinated Universal Time}}
  \acro{VIMS}{\textit{Vehicle IoT Metrics System}}
  \acro{VIN}{\textit{Vehicle Identification Number}}
  \acro{WCET}{\textit{Worst Case Execution Time}}
\end{acronym}

\begin{itemize}
  \item \ac{ADAS} -- sistema de control de un vehículo que utiliza sensores del entorno
        (radar, láser, visión por computador) para mejorar la seguridad al volante y
        la seguridad del entorno ayudando a los conductores a reconocer y reaccionar
        ante eventos o situaciones de tráfico potencialmente peligrosas \cite{hermawanAcquisitionModelingEvaluating2020}.
  \item \ac{API} -- conjunto de definiciones, subrutinas y protocolos que ofrecen
        ciertos \textit{softwares} para ser usados por otras aplicaciones como
        capa de abstracción sobre el original \cite{InterfazProgramacionAplicaciones2021}.
  \item \ac{BLE} -- tecnología \ac{PAN} que permite la comunicación entre dispositivos
        con un rango similar a Bluetooth pero un menor consumo de energía.
  \item \ac{BR/EDR} -- terminología asociada al estándar Bluetooth que define el modo
        de funcionamiento clásico del mismo. Se caracteriza por contar con 79 canales
        de $1~MHz$ de ancho de banda, operar en la red de $2.4~GHz$, ser \ac{P2P},
        transmitir a una velocidad máxima de $3~Mbps$ y tener un consumo medio de
        $1~W$ \cite{ComparisonBluetoothBR}.
  \item Bus \ac{CAN} -- protocolo de comunicaciones de tiempo real para el envío de
        mensajes en entornos distribuidos, permitiendo la comunicación
        entre múltiples CPUs.
  \item \ac{CRUD} -- funciones básicas que se realizan sobre una base de datos y que
        conforman el conjunto mínimo de operaciones que se deben poder realizar sobre
        una.
  \item \ac{DMS} -- sistema de ordenación por prioridades en donde se asignan valores
        de prioridad a las tareas según su plazo o \textit{deadline}, de forma que cuanto
        menor es su plazo mayor será su prioridad.
  \item \ac{DTC} -- códigos de diagnóstico almacenados en un vehículo que identifican
        un error que tiene el vehículo o que ha tenido.
  \item \ac{ECU} -- centralita electrónica conectado a todos los sensores y sistemas
        del vehículo que recibe la información de los mismos y ejecuta comandos o acciones
        contra ellos.
  \item \ac{GB} -- unidad de almacenamiento de información equivalente a $10^9$ bytes.
  \item \ac{GPIO} -- pin genérico cuyo comportamiento puede ser controlado en tiempo
        de ejecución.
  \item \ac{GPS} -- sistema que permite posicionar cualquier objeto con una 
        precisión de hasta centímetros usando cuatro o más satélites y 
        trilateración \cite{GPS2021}.
  \item \ac{GUI} -- conjunto de componentes visuales que permiten a los usuarios
        interactuar con una aplicación de manera visual e intuitiva.
  \item \ac{ICPP} -- protocolo de inversión de prioridades que se usa cuando múltiples
        tareas intentan acceder al mismo recurso compartido en donde la tarea $t_i$ que
        accede al recurso hereda una prioridad de techo equivalente a la de la tarea
        de mayor prioridad que accede al recurso. Por ejemplo, si las tareas $t_1$ y
        $t_2$ con prioridades $10$ y $5$ respectivamente acceden al recursos $r_1$, la
        prioridad de la tarea $t_2$ será de $10$ mientras acceda a dicho recurso.
  \item \ac{IoT} -- concepto que se refiere a la interconexión digital de objetos 
        cotidianos con Internet \cite{InternetCosas2021}.
  \item \textit{jitter} -- retardo relativo que se produce en las comunicaciones
        y que afecta directamente a la saturación de la red y a la capacidad de
        transmisión de la misma.
  \item \ac{LTE} -- también conocido como ``\texttt{4G}'', es la cuarta generación
        del estándar de conexión móvil a Internet que mejora a su predecesor, el \texttt{3G}.
  \item \ac{MSB} -- bit, de acuerdo con su posición, tiene mayor valor (extremo izquierdo).
  \item \ac{NTP} -- protocolo de sincronización temporal que funciona sobre UDP usando para mantener
        los relojes de los sistemas informáticos en base al reloj atómico universal, que
        es \textit{stratum 1}.
  \item \ac{OBD} -- sistema de diagnóstico a bordo de vehículos que cuenta con múltiples estándares según la región de uso. Estos
        sistemas ofrecen una monitorización activa y control completo
        sobre el motor y otros dispositivos del vehículo \cite{OBD2021}.
  \item \ac{P2P} -- protocolo de comunicación entre dos nodos en donde sendos nodos se
        interconectan entre sí para el intercambio de información sin necesidad de usar
        un intermediario.
  \item \ac{PAN} -- redes destinadas a la comunicación entre dispositivos en una
        misma red o malla. Tiene un alcance muy limitado, de unos pocos metros por
        lo general.
  \item \ac{PID} -- parámetro utilizado en la comunicación mediante \ac{OBD}--II que
        identifica el tipo de petición que el mecánico/usuario quiera hacer al sistema.
        Son valores enteros de 16 bits que funcionan en formato clave--valor. La información
        devuelta por el conector es también un conjunto de bytes de 32 bits en total que
        representan la información solicitada por el usuario. Por lo general, se deben
        realizar ciertas operaciones aritméticas con ese conjunto de bytes para obtener
        la representación decimal del valor \cite{OBDIIPIDs2021}.
  \item \ac{PWM} -- señal cuadrada de periodo habitualmente constante, entre flancos de
        subida, en la que se modula el tiempo a nivel alto.
  \item \ac{REST} -- estilo de arquitectura \textit{software} para sistemas hipermedia
        distribuidos, como la World Wide Web. En la actualidad se usa en sentido más
        amplio para describir cualquier interfaz entre sistemas que usa directamente
        HTTP para obtener datos o indicar una operación sobre un conjunto de datos
        en cualquier formato (aunque habitualmente se usa JSON o XML) \cite{TransferenciaEstadoRepresentacional2022}.
  \item \textit{Smog} -- nube baja formada por dióxido de carbono, hollín, humo y polvo
        en suspensión que se forma sobre las grandes ciudades o núcleos industriales.
  \item \ac{SoC} -- denominación asociada a los sistemas (generalmente embebidos) que
        aúnan todos los módulos que requieren para funcionar (y otros tantos para añadir más
        funcionalidades) en un único \textit{chip} que expone al exterior dichas
        funcionalidades mediante una interfaz \textit{hardware} como si fuese un único
        circuito integrado.
  \item trilateración -- método matemático que permite determinar las posiciones
        relativas de objetos usando la geometría de los triángulos \cite{Trilateracion2021}.
  \item \ac{UTC} -- principal estándar de tiempo por el cual el mundo regula los relojes y el tiempo.
        Entre otras características, es un reloj monotónico (no puede volver al pasado).
  \item \ac{VIN} -- código único, incluyendo el número de serie, que identifica individualmente
        los vehículos a motor, a remolque, motocicletas, \textit{scooters} y \textit{mopeds},
        tal como se define en las ISO 3779 (contenido y estructura) e ISO 4030 (ubicación
        y acoplamiento) \cite{VehicleIdentificationNumber2022}.
\end{itemize}
