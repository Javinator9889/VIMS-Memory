\begin{abstract}
  En este documento se va a tratar el diseño y especificación de \ac{VIMS},
  un proyecto de ingeniería que modela, diseña y arquitecta un sistema
  conformado por un dispositivo embebido y un servidor en la nube. El
  dispositivo se conecta al vehículo y transmite, mediante redes inalámbricas,
  los datos al servidor, que hará todo el procesamiento y gestión.
  El objetivo principal es devolverle al propietario del vehículo la
  sensación de control sobre el mismo,
  accediendo fácilmente a los datos recogidos por el automóvil así como
  de los mensajes de error.

  Para ello, primero se recogerán datos de una muestra de conductores y
  no conductores para identificar correctamente las necesidades de los
  usuarios y ajustar mejor el producto.
  
  A continuación, se elicitarán los requisitos que permitirán posteriormente
  modelar y diseñar el sistema de forma fiel.

  La fase anterior permite trabajar directamente en los diagramas que modelan
  el sistema, tanto a nivel de casuísticas como en qué estructura deberá tener.
  Esto es fundamental porque simplificará y acotará las etapas de desarrollo
  posteriores.

  Además, se trabajará en el estudio del modelo matemático que permite traducir
  los datos recibidos por los vehículos según el estándar \ac{OBD}--II. A su vez,
  se estudiarán las características del sistema \textit{hardware} lo que permitirá
  desarrollar y construir una placa de control que será la encargada de gestionar los
  parámetros del dispositivo \ac{VIMS}.

  Por último, se estudian las distintas tareas que compondrán el sistema \ac{VIMS}
  mediante el análisis de tiempo de respuesta de las mismas. Dicho análisis
  determina la planificabilidad del sistema y asegura un correcto y predecible
  funcionamiento en cualquier circunstancia.
\end{abstract}

\selectlanguage{english}
\begin{abstract}
  \ac{VIMS} design and specification project is an integral engineering development
  which designs, models and architects a whole system built with an embedded
  device and a remote cloud server. The device is attached to the driver's vehicle
  and by using wireless communication transmits the data to the remote server,
  which handles the entire processing and data generation.
  The main objective is to bring the driver's sensation back of having control
  over the vehicle, easily accessing to the entire information it provides
  alongside error messages in an easy, accessible way.

  Firstly, a quest will be done so relevant data is collected from a sample
  of drivers and non-drivers which will help identifying their needs and
  better adjusting the final product.

  Then, the requirements will be elicited which will allow the modeling
  and the design of the system in further steps of the development.

  The previous step allows working directly with the diagrams that will
  model the system in both behavioral and structural manners. This is
  crucial as it will simplify and delimit the next steps of the project.

  In addition, there will be a mathematical analysis on how to translate
  the data received from the vehicle itself into human-readable information,
  based on the \ac{OBD}--II standard. Furthermore, hardware characteristics
  will be studied which will allow developing and building a PCB which will
  handle the parameters of the \ac{VIMS} device.

  Finally, a task study will be done for \ac{VIMS} system. This will lead us
  through the definition of the tasks themselves as well as the response time
  analysis for the whole system. Thus the analysis determines if the entire
  system can be scheduled asserting a well-known, correct behavior under any
  circumstances.
\end{abstract}

\selectlanguage{spanish}
