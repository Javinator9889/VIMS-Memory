A raíz del apartado anterior y de la cantidad de datos que se pretenden recoger
durante el funcionamiento del sistema, una parte fundamental es la planificación
de las tareas del mismo.

En principio, \ac{VIMS} no se define como un sistema en tiempo real. Sin embargo,
la envergadura del proyecto y la cantidad de acciones a realizar de forma coordinada
y concurrente requiere de una planificación del proyecto a la hora de definir tareas,
plazos y periodos.

En un primer análisis, se detectan las siguientes casuísticas que llevarán asociada
la tarea pertinente:

\begin{enumerate}
  \item Lectura de datos desde el conector \ac{OBD}--II. Esta tarea almacenará la
        información en crudo del conector en una estructura de datos propia, preparada
        para que otra tarea los adapte y transforme en información entendible. Además,
        establecerá una marca temporal para los datos de manera que se sepa exactamente
        en qué momento se obtuvieron.
  \item Conversión de los datos leídos desde el \ac{OBD}--II. Esta tarea deberá
        leer y convertir los datos almacenados en crudo por el \ac{OBD}--II en
        datos legibles y entendibles por el ser humando. Se encargará además de
        definir las unidades de medida asociadas al dato en sí.
  \item Transmisión de los datos al servidor en la nube. Esta tarea deberá enviar
        los datos ya preparados con su unidad correspondiente y su marca temporal
        usando el protocolo MQTT al servidor en la nube. Según la tecnología de
        recepción presente en el \textit{backend}, se pueden enviar los datos
        individualmente a un \textit{endpoint} específico de MQTT (lo cual simplificaría
        la gestión de la información por parte del servidor) o se deberán enviar todos
        en conjunto como un JSON al servidor en backend.
  \item Persistencia de los datos en la tarjeta microSD. Esta tarea almacenará
        los datos provisionalmente en la tarjeta microSD si no han podido ser enviados
        al servidor remoto (por ejemplo, por no disponibilidad de la red). Se deberán
        serializar los datos junto con su marca temporal para una posterior retransmisión
        cuando haya disponibilidad de red.
  \item Estado de la conectividad de red. Esta tarea se encargará de comprobar
        periódicamente que la conexión de red está disponible y es accesible desde
        el dispositivo, simplificando la lógica de comunicación del resto de tareas.
  \item Sincronización del reloj con la red. Esta tarea se encargará, mediante el
        protocolo \ac{NTP}, de mantener el reloj interno de la placa sincronizado con el
        tiempo \ac{UTC}, de forma que todas las marcas temporales asociadas están siempre
        con la hora adecuada.
  \item Ubicación del dispositivo. Esta tarea se encargará de actualizar la ubicación
        del dispositivo y almacenar los datos para su posterior envío al servidor remoto.
  \item Estado eléctrico del sistema. Esta tarea se encargará de comprobar periódicamente
        la tensión de entrada del sistema para detectar cambios drásticos en dicha
        tensión (de $\approx \pm 1.1V$) para saber cuándo el sistema debe entrar en
        modo de bajo consumo.
\end{enumerate}

Las tareas anteriores deberán planificarse siguiendo un esquema de planificación de
tiempo real que contemple frecuencia de activación, retardos relativos producidos
por la interdependencia de las tareas, tiempo de cómputo estimado, prioridades
según frecuencia, acceso a recursos compartidos y herencia de prioridad por acceso
a dichos recursos (prioridad de techo).
