El desarrollo del sistema \ac{VIMS} es un proceso multidisciplinar en el que se deben
desarrollar varias áreas de conocimiento. Este proyecto se ha postulado como
un desarrollo integral de ingeniería y es por eso por lo que está dividido en
varios bloques que conforman un factor clave en el desarrollo del mismo.

Estas secciones están bien diferenciadas y se conforman de: un estudio de mercado y de
las características de los usuarios, un estudio matemático asociado a la lectura de
valores y adecuación del \textit{hardware}, el proceso de diseño \textit{hardware}
en sí, el diseño \textit{software} del sistema y el análisis de planificación del mismo.

\begin{itemize}
  \item El estudio de mercado pretende averiguar y formalizar las necesidades de los
  conductores y usuarios de la vía. Es la primera aproximación y facilita la
  delimitación del producto y, sobre todo, ofrecerle al usuario final algo de utilidad
  y que pueda necesitar.
  \item El estudio matemático se encarga de investigar la ``traducción'' de los
  valores recibidos por el vehículo (según los datos asociados al estudio de mercado
  realizado con anterioridad).
  \item El diseño \textit{software} modela principalmente cómo se va a estructurar
  el sistema y cómo debe comportarse ante los distintos eventos que puede recibir.
  Esta fase conlleva realizar diagramas lógicos y de diseño del sistema en su conjunto.
  \item El diseño \textit{hardware} conlleva tanto el estudio de los componentes del
  sistema así como de las restricciones físicas del mismo.
  \item El análisis de planificabilidad complementa el diseño \textit{software}
  y estudia si el sistema es planificable. En los requisitos no se define \ac{VIMS}
  como un sistema en tiempo real, pero la cantidad de componentes que contiene y las
  acciones que tiene que realizar requieren del uso de subrutinas y de una planificación
  previa para asegurar un correcto funcionamiento del mismo.
\end{itemize}

Es importante detallar que pese a que el sistema se compone de varios componentes,
son dos los principales que lo caracterizan:

\begin{enumerate}
  \item La placa, \ac{VIMS}, que va embebida en los vehículos del sistema. Se encarga
  de toda la lectura, adaptación y emisión de datos. Además, cuenta con soporte para
  poder realizar una transmisión de la información a un dispositivo asociado mediante
  redes \ac{PAN}.
  \item El servidor \textit{cloud}, el ``cerebro'' encargado de recibir las tramas,
  los datos y la información relativa a las placas \ac{VIMS}, los dispositivos de
  usuario y demás componentes. Además, tiene la responsabilidad de ofrecer a los
  usuarios una \ac{GUI}, generar información relevante a partir de los datos (como
  estadísticas), gestionar las suscripciones y enviar periódicamente la información
  al usuario.
\end{enumerate}
