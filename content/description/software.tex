El proyecto cuenta con una parte \textit{software} muy importante, ya que está
conformado por múltiples sistemas que deben cooperar entre sí. Por una parte,
el desarrollo del \textit{software} del proyecto aborda los siguientes aspectos:

\begin{itemize}
  \item Desarrollo de la aplicación embebida que irá en el dispositivo enganchado
        al coche en sí. En dicho dispositivo se programará, entre otros, la planificación
        del sistema según el análisis de planificabilidad llevado a cabo en el punto
        \ref{sec:rt-design}, la gestión de los datos y de la información y la transmisión
        al servidor remoto.
  \item Desarrollo de la infraestructura en la nube que alojará la lógica de almacenamiento de
        los datos y de la información generada a partir de ellos para los usuarios de
        dispositivos \ac{VIMS}.
  \item Desarrollo de la aplicación web que servirá de pasarela entre los usuarios
        y sus datos alojados en el servidor en la nube, mediante una interfaz de usuario
        que facilitará ciertas tareas de gestión y la observación de los datos.
  \item Desarrollo de la aplicación móvil que permitirá la conexión directa con la
        placa \ac{VIMS} para su configuración inicial y la visualización de la información
        en el momento.
\end{itemize}

El sistema sigue el prototipo de una arquitectura cliente--servidor, con uno o varios
clientes que se conectan a un servidor en la nube que realiza tareas de gestión de
la información, entre otros.

Uno de los requisitos principales de la aplicación (que se vio además en el estudio
del mercado) era que esta fuese accesible, simple y fácil de entender. Mediante la
definición de tanto la aplicación web como la aplicación móvil se pretende unificar
el acceso a la información de los usuarios y sus vehículos. Con ambas aplicaciones,
se busca que, de un vistazo, se tenga acceso a los últimos viajes, estadísticas,
perfiles de conducción y demás.

Además, la arquitectura del servidor permite generar notificaciones personalizadas
cuando sucedan ciertos eventos. Por ejemplo, se pueden enviar correos electrónicos
con información sobre el último viaje, cuánto ha durado el depósito de gasolina, fecha
recomendada para el siguiente mantenimiento, revisión del estado de los neumáticos,
etc. De esta manera, se pretende que el usuario no esté directamente pendiente del
estado mecánico de su vehículo sino que delegue esa tarea en el sistema \ac{VIMS}.

Por su lado, la programación del dispositivo \ac{VIMS} en sí es una parte crítica y
fundamental en el desarrollo del proyecto. El dispositivo va a estar en un entorno
no controlado con condiciones cambiantes, sobre todo en lo referente a las comunicaciones.
Es fundamental que \textit{software} del dispositivo embebido
sea resiliente y esté diseñado para soportar situaciones adversas en lo referente
a la transmisión de la información.

Esto se detalla más adelante en los requisitos, y algunas de las características que
se necesitan tener en el dispositivo en sí son el almacenamiento provisional de los
datos, la retransmisión de la información en caso de fallo, modos de bajo consumo
cuando el vehículo no está en marcha, etc.
