Para el desarrollo de este proyecto, se hizo un estudio de mercado tanto de los
consumidores como de sus características, además de una evaluación exhaustiva
de qué les gustaría tener en su vehículo.

Este proyecto pretende en un futuro salir a mercado y suplir características que los
usuarios echan en falta en sus correspondientes medios de transporte. Como en principio
funciona con cualquier vehículo que cuente con \ac{OBD}--II, las respuestas no se han
limitado a aquellos conductores que condujesen turismos sino cualquier tipo de
automóvil: motocicleta, camión, etc.

Es importante destacar que el estudio tiene varios sesgos que han restringido
y delimitado las respuestas que se han registrado:

\begin{enumerate}
  \item Se ha realizado un cuestionario usando Google Forms, una plataforma de Google
        que permite preparar una serie de preguntas y respuestas y aplicar ciertos
        filtros sobre ellas. Por ejemplo, para aquellos que dijeron ser conductores,
        se hicieron preguntas diferentes frente a quienes no lo fueran.

        Esto permite obtener datos más fidedignos y acotados según la población que
        respondiera. Sin embargo, tiene una limitación implícita: restringe el acceso
        a aquellos con conocimientos ``suficientes'' acerca de la plataforma. Pese
        a que el producto pretende ser lo más accesible posible, no hy que olvidar
        que este tipo de tecnologías permanecen desconocidas para una gran parte
        de la población con escasos conocimientos acerca de Internet o de las
        nuevas tecnologías. Se comentará más adelante, pero esto se ve reflejado
        principalmente en la edad media de quienes respondieron el cuestionario.

        Por otra parte, al ser un cuestionario aparece otra limitación implícita
        y es la validación y verificación de las respuestas.
\end{enumerate}
