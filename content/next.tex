En este apartado se enuncia el trabajo que queda por realizar y distintas mejoras
que pueden mejorar el desarrollo del producto o incrementar su valor final.

Este conjunto de ideas pendientes nace directamente de aquellas tareas o fases
que no se han podido implementar, pero también incluye ideas que han ido
surgiendo durante el desarrollo del proyecto.

En lo referente al trabajo pendiente, quedaría:

\begin{itemize}
  \item Completar el desarrollo del \textit{software} del dispositivo \ac{VIMS}.
        Con el análisis del tiempo de respuesta debidamente realizado, se
        puede completar la programación de la placa definiendo las tareas,
        recursos compartidos y ajustándose a los tiempos estimados.
  \item Realizar pruebas físicas con el dispositivo en sí, conectándolo a un
        conector \ac{OBD}--II y comprobar que los datos leídos se corresponden
        con los esperados.
  \item Definir la \ac{API} de comunicación contra el servidor, con las operaciones
        pertinentes según los diagramas de casos de uso y el diagrama de bloques
        general del proyecto.
  \item Definir un modelo de datos a almacenar. Lo más eficiente parece ser una
        serie temporal y transmitir los datos vía JSON en un único tópico de
        MQTT. Esto permite reducir las comunicaciones (solo se realiza una),
        aumentar el conjunto de datos de forma dinámica, sin tener que añadir
        ningún tópico MQTT nuevo y realizar el envío comprimido en un estándar
        entendible por cualquier lenguaje de programación actual.
  \item Desarrollar la plataforma web de interacción con el usuario. Actualmente,
        se cuenta ya con un prototipo de muestra de datos, por lo que sería necesario
        completar con la web que manejará toda la información y las interacciones.
  \item Definir el modelo de generación de datos e información según se van recibiendo
        por parte de los dispositivos \ac{VIMS} para enviar datos personalizados
        a los participantes.
  \item Desarrollar la aplicación móvil que se conectará directamente a los
        dispositivos \ac{VIMS} para monitorizar la información del vehículo
        mediante \ac{BLE}.
\end{itemize}

Por otra parte, en lo referente a las mejoras, se contempla:

\begin{itemize}
  \item Ampliar el espectro de muestra de los encuestados cambiando la dinámica
        o la forma de obtener información. Se sugiere realizar entrevistas,
        dinámicas supervisadas, etc. Esto es importante para no excluir de la
        muestra a potenciales candidatos por edad, restricciones físicas, etc.
  \item Desarrollar una aplicación web progresiva que sustituya a la interfaz
        web y la aplicación móvil. Con esto se conseguiría realizar un único
        desarrollo para todas las plataformas disponibles.
  \item Introducir información relevante sobre los códigos de error disponibles.
        En la web \url{https://www.obd-codes.com/trouble_codes/} se da muchísima
        información sobre las causas de un error, signos evidentes del error y
        cómo solucionarlo. Resultaría de gran interés incluir esa información
        directamente en el análisis del estado del vehículo.
  \item Incluir códigos \ac{PID} propietarios, además de los estándar. Usando
        la información del \ac{VIN} por ejemplo, obtener el vehículo en que se
        ha instalado \ac{VIMS} e intentar hacer uso de códigos \ac{PID} propietarios
        para acceder a más información que se pueda obtener del vehículo.
  \item Ofrecerle al usuario la posibilidad de obtener datos según sus propios
        \ac{PID}s. Muy en la lógica de la idea anterior, si un usuario cuenta
        con sus propios códigos \ac{PID} ofrecerle la posibilidad de que los
        introduzca y que genere información a partir de esos datos.
\end{itemize}
