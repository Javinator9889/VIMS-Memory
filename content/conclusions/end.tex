Este proyecto se inició como una idea que tenía en mente para
hacer como \textit{hobbie} personal. Tengo la suerte de que
la Ingeniería que he estudiado es también una de las cosas que
hago a diario, tanto por trabajo como por mera diversión.

Lo que he estudiado en el Máster de Software de Sistemas Distribuidos y
Empotrados me animó a materializar esta idea (al igual que he hecho con
tantas otras) y hacerla un producto real que pueda ayudar a la gente
real. He puesto toda la carne en el asador y este es el resultado.

Considero que este es el primer paso en la dirección correcta. Soy
consciente de que es un proyecto muy ambicioso, y también de que
se me han juntado un montón de cosas que han afectado directamente
en el desarrollo del mismo. No obstante, tengo mucha ilusión y
espero a finales de año estar haciendo las primeras pruebas en
un vehículo real, esta vez ya de nuevo como un \textit{hobbie}.

Aplicar el proceso científico (y en particular, el de ingeniería)
nos dota de una visión global del conjunto de tareas y labores
a realizar, más o menos completa pero por lo general acertada. Esto
nos permite estimar plazos, tiempos, costes y presupuestos. Considero
que la estimación inicial realizada es bastante realista y que
las causas por las cuales no se ha podido cumplir son ajenas a la
planificación en sí. Es importante no olvidar que la planificación
se hizo con la mentalidad de dedicarse a tiempo completo al proyecto
(y ahí veo ``resquicios'' del PFG, al cual le dedicamos una media de
6 u 8 horas diarias). Sin embargo, las etapas por las que uno va
pasando determinan directamente el resultado del trabajo así como
que las horas del día limitan directamente lo que puedes hacer.

Creo que este proyecto puede ayudar directamente a mucha gente,
acercar el mundo del automovilismo a más personas y disipar esa
niebla difusa que hace que parezca que es algo para eruditos.
Como en muchas otras cosas, los Ingenieros tenemos en nuestra
mano el poder de cambiar el mundo. Y esto nunca hay que hacerlo
solos, sino acompañándonos de esas personas que compartan nuestra
ambición y cumplimenten nuestro saber, que es muy escaso. Por eso,
espero seguir trabajando en este proyecto durante mucho más tiempo,
rodeándome de gente que quiera colaborar y traer esta idea a la
vida.

No considero recortar el contenido de este PFM un fracaso sino
un primer paso en la dirección adecuada. Considero que una buena
especificación garantiza directamente el éxito del trabajo, o al
menos delimita las posibilidades de fallo. Por ello, no tengo otra
cosa que decir sino que ha sido un éxito y una gran forma de completar
mi etapa universitaria.

\begin{figure}[H]
  \centering
  \includegraphics[width=\linewidth]{images/the-end.jpeg}
\end{figure}
