Para realizar este proyecto, se ha tenido que indagar en las siguientes áreas de
conocimiento:

\begin{itemize}
  \item Proceso de elicitación de requisitos, revisión y validación.
  \item Diseño esquemático y de abstracción de un sistema complejo,
        teniendo en cuenta las distintas variables que lo componen.
  \item Planteamiento de los casos de uso de un proyecto para
        su delimitación y funcionamiento esperado.
  \item Análisis estadístico de una muestra variada de cara a extraer
        conclusiones adecuadas y ciertas sobre la población y los
        datos que se obtienen.
  \item Identificación de las necesidades de los \textit{stakeholders},
        así como la identificación de los \textit{stakeholders} en sí.
  \item Diseño y desarrollo de componentes para KiCad cuando la librería
        estándar no los contiene, siguiendo las restricciones lógicas del
        componente en sí así como sus dimensiones físicas.
  \item Diseño y desarrollo de PCBs que engloben la funcionalidad del
        sistema y lo doten del funcionamiento esperado.
  \item Mayor comprensión y conocimiento sobre los protocolos de
        comunicación de un vehículo, en particular, del bus \ac{CAN}.
  \item Conocimiento en arquitecturas en la nube, proveedores \textit{cloud}
        y soluciones ya existentes que puedan facilitar el trabajo, como
        Grafana.
  \item Comunicaciones de red de bajo nivel, como \ac{LTE} o \ac{BLE}.
  \item Análisis y comprensión de las tareas en tiempo real que definen
        un sistema. Acceso compartido a recursos y dependencias temporales
        entre ellas.
  \item Capacidad de adaptación al cambio y de afrontar decisiones complicadas,
        como la de convertir el proyecto a un proyecto de especificación.
\end{itemize}
