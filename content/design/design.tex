En esta sección se detalla cómo se ha diseñado en general el sistema. Por una parte,
en la sección \ref{sec:sys-diagrams} se introducen los diagramas de casos de uso que dan
una visión general de cómo se va a organizar todo el sistema. Además, se comentan
también los diagramas de bloques que dan forma a la arquitectura final que se va
a implementar. A continuación, en la sección \ref{sec:hardware-design} se introduce
la PCB diseñada y las distintas decisiones de diseño que se han considerado a la hora
de la fabricación. Después, en la sección \ref{sec:3d-design} se habla del proceso
de diseño 3D y de cómo el sistema se imagina estará embebido dentro del vehículo.
Luego en la sección \ref{sec:rt-design} se estructurarán
las tareas definidas en la sección \ref{sec:rt-analysis} y se realizará la
planificación en tiempo real del sistema. Finalmente, se comenta en más detalle
el proceso de diseño \textit{software}, en la sección \ref{sec:software-design},
en donde se comentarán los casos de uso del sistema, cómo se ha diseñado el código
que va en \ac{VIMS} y diseños y patrones para las interfaces gráficas de la
implementación final.
