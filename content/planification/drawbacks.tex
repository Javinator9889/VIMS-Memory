El proyecto ha contado con múltiples retrasos durante su realización, empezando
desde el año pasado: hubo un problema con la matriculación y se retrasó todo
un año, debido a las diversas gestiones académicas.

Durante el mes de junio de 2021 se realizó una primera especificación de
requisitos, la cual fue revisada por el tutor. El trabajo se retomó nuevamente
en el mes de febrero, con el comienzo del nuevo curso escolar. Los
motivos por los cuales no se trabajó durante el verano y los últimos meses de 2021 fueron:

\begin{itemize}
  \item Se finalizó un trabajo que se estaba llevando a cabo de profesor de Grado Superior,
        junto con otros cursos que se impartieron en otros institutos.
  \item Se empezó a trabajar en Teldat en el mes de septiembre, comenzando la jornada
        completa de 8h diarias. Los primeros meses fueron de adaptación ya que coincidió
        que fue el primer trabajo de esta índole.
  \item Se empezó un máster mientras se empezó el trabajo en Teldat (adicional al
        Máster de Software en Sistemas Distribuidos y Empotrados) que se ha extendido
        hasta el mes de junio.
  \item Se estuvo trabajando como \textit{freelance} para Saturno Labs, realizando
        diversos proyectos durante esos meses hasta finales de enero.
\end{itemize}

Durante el mes de febrero se revisaron los requisitos y se comenzó el diseño de diagramas
que modelarían el sistema. Se seguía con el trabajo a jornada completa así como el
máster, por lo que solo se podía dedicar cierto tiempo por las tardes.

En los meses siguientes, la intensidad del trabajo se vio disminuida porque fue una
época de más cantidad de clases en el máster (ya que los días no son fijos sino
que van variando) y de mayor intensidad laboral. Esto se tradujo en que el tiempo
que se dedicaba por las tardes era menos eficaz de lo habitual.

En el mes de mayo se inició una mudanza que se completó definitivamente en la
segunda semana del mes de junio. Esta mudanza se tradujo directamente en no poder
trabajar en el PFM, ya que las tardes se emplearon en mover cajas y establecerlo
todo.

Por último, lo restante del mes de junio y principios de julio se dedicó entre otras
cosas a ultimas los detalles del proyecto y realizar el planteamiento teórico del mismo.
Como las clases del máster finalizaron en el mes de mayo, se pudo avanzar a mejor
ritmo y adelantar ciertos elementos fundamentales, como el análisis del tiempo de
respuesta y escribir la memoria.

Se muestra una línea temporal resumen de la situación expuesta anteriormente:

\begin{vtimeline}[.85]{Línea temporal de la evolución del proyecto.}
  01/06/2021 & Se lanza el cuestionario para conocer las necesidades de los conductores. \\
  28/06/2021 & Primera elicitación de requisitos con sus correspondientes revisiones %
               por parte del tutor. Se dejó una versión cerrada de los requisitos que %
               modelarían el proyecto que hoy se expone. \\
  30/07/2021 & Se finaliza el trabajo como profesor de Grado Superior así como los %
               otros cursos complementarios que se estaban impartiendo. \\
  03/09/2021 & Se empieza a trabajar en Teldat a jornada completa. \\
  20/09/2021 & Se comienza el Máster de Liderazgo Internacional de Ingeniería y Arquitectura, %
               de la UPM junto con la Fundación Rafael del Pino. \\
  15/10/2021 & Se empiezan a desarrollar proyectos como \textit{freelance} para %
               Saturno Labs. \\
  30/01/2022 & Se finalizan los proyectos en desarrollo con Saturno Labs. \\
  02/02/2022 & Se retoma el desarrollo del PFM. Se trabaja activamente en los diagramas %
               que modelan el sistema y en los casos de uso. \\
  28/02/2022 & Se da por concluida la fase de modelado del sistema. Se comienza a trabajar %
               en el diseño lógico de la PCB del sistema. \\
  15/03/2022 & Se completa y verifica el diseño lógico y físico del componente diseñado %
               a mano que representa el dispositivo T-SIM7000G. \\
  05/04/2022 & Llega el pedido de dispositivos T-SIM7000G. \\
  15/04/2022 & Se completa el diseño lógico y físico de la PCB del dispositivo \ac{VIMS}. %
               Se comienza con las pruebas de la placa T-SIM7000G. \\
  05/05/2022 & Se reciben las PCBs (5) fabricadas por JLCPCB. Se comienza el desarrollo del %
               \textit{software} del dispositivo \ac{VIMS}. \\
  25/05/2022 & Se establecen las tareas que conformarán el dispositivo \ac{VIMS} y se %
               comienza el análisis del tiempo de respuesta. \\
  01/06/2022 & Se inicia el proceso de mudanza. \\
  15/06/2022 & Se finaliza el proceso de mudanza. Se retoman las tareas del PFM. \\
  25/06/2022 & Se finaliza el análisis del tiempo de respuesta y se escriben las conclusiones. %
               Comienza la escritura de la memoria. \\
  13/07/2022 & Entrega de la memoria. \\
\end{vtimeline}

Tras este análisis, el tiempo que ha llevado el proyecto en total sería de un año.
Teniendo en cuenta las situaciones descritas anteriormente, se estima la duración
del proyecto real en aproximadamente 1440 horas.
