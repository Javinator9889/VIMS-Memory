El proyecto que se pretende desarrollar es un proyecto de ingeniería. Esto se
traduce en que la metodología y la forma de trabajo son pilares fundamentales
en el desarrollo del mismo.

El primer paso realizado fue el de recopilar recursos e información de qué
necesitaban específicamente los usuarios. Para ello, se elaboró un cuestionario
en donde de forma general se preguntaba a conductores y no conductores qué querrían
tener en su vehículo. Los primeros sirvieron de grupo de control, los segundos para
aumentar la entropía de los datos obtenidos. El primer análisis realizado se detalla
en la sección \ref{ssec:user-req}, de la especificación de requisitos. Posteriormente,
en el punto \ref{chap:merch} se analiza en mayor profundidad los datos obtenidos
y se extraerán conclusiones.

A continuación, se realizó un estudio sobre qué plataformas y dispositivos están
accesibles de forma global para la generación y transmisión de datos. Esta fase
se centró principalmente en ``descubrir'' variantes del modelo ESP32 que incluyesen
ciertas antenas para permitir una mayor conectividad. Tras valorar diversas opciones,
se decidió usar el LILYGO T-SIM7000G ESP32 que incluye soporte de forma nativa para
tarjetas microSD, antena \ac{LTE}, alimentación externa por batería, antena \ac{GPS}, WiFi y
Bluetooth.

Una vez se decidió que dispositivo físico se iba a utilizar, se comenzó con el desarrollo
de los distintos diagramas que modelan el sistema, tanto lógicos como de diseño. Esta
parte fue crucial para asentar las bases de lo que será el proyecto y ha permitido seguir
el avance del mismo.

Por último, se realizó el serigrafiado de la placa y se comenzó la implementación
física de los diseños realizados. Sin embargo, esta etapa no se ha podido completar
por distintos contratiempos que se comentan en más detalle en el punto \ref{chap:planification}.
